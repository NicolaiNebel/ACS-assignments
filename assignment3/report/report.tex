\documentclass[11pt]{article}

\usepackage[utf8]{inputenc}
\usepackage[T1]{fontenc}
\usepackage[head=26pt, a4paper, margin=1.2in, top=1.4in, bottom=1.75in]{geometry}
\usepackage{fancyhdr}
\usepackage{lastpage}
\usepackage[hidelinks, colorlinks, urlcolor=blue, linkcolor=black,citecolor=magenta]
{hyperref}
\usepackage{amsmath}
\usepackage{amsthm}
\usepackage{amssymb}
\usepackage{graphicx}
\usepackage{float}
\usepackage{listings}
\usepackage{mathtools}
\usepackage{enumitem}


\usepackage{indentfirst}
\usepackage{a4wide}
\usepackage{color}
\usepackage{lipsum}
\usepackage{multicol}
\usepackage{tikz}
\usetikzlibrary{arrows,shapes,positioning}

% ---------------- Page and margin/header/footer Setup -----------------
\pagestyle{fancy}
%\fancyhf{} % Clears header and footer
\fancyhead{}
\fancyfoot{}
\lhead{ACS --- Assignment 3}
\rhead{DIKU}
\lfoot{Page \thepage\ of \pageref{LastPage}}
\rfoot{Nicolai Jørgensen \\ Yiran Zhang}
\renewcommand{\headrulewidth}{0.4pt}
\renewcommand{\footrulewidth}{0.4pt}
% ----------------------------------------------------------------------

\newtheorem{mythm}{Theorem}
\newtheorem{mydef}{Definition}

\DeclarePairedDelimiter{\ceil}{\lceil}{\rceil}
\newcommand\numberthis{\addtocounter{equation}{1}\tag{\theequation}}

\newcommand{\HRule}{\rule{\linewidth}{0.5mm}}

\title          {Assignment 3}
\author         {Nicolai Jørgensen and Yiran Zhang}

\begin{document}

\maketitle
\newpage

\section{Question 1}
\begin{enumerate}
	\item
	In a system implementing force and no-steal, it is not necessary to implement a scheme for redo. Because all committed transaction to a page have been written to disk at the time of a subsequent crash. And it is also not necessary for undo, since all dirty write have not been written to disk at the time of a subsequent crash.
	\item
	 Non-volatile storage retains data even when power goes off. While the information in stable storage is certainly not lost(theoretically). A natural catastrophe may result in a loss if not the probability of data loss is negligible. And the non-volatile is faster than stable storage in terms of data access time.
	 
	 Crash failure can be survived by non-volatile storage, media failure can be survived by stable storage.
	 \item
	 When a transaction is committed, the log tail is forced to stable storage. Or when pages are written to disk in yet uncommitted transactions, the log tail is forced to stable storage.

	 For the first case, if a transaction made a change and committed, the no-force approach means that some of the changes may not have been written to disk at the time of subsequent crash. Without a record of these changes, there would be no way to ensure that the changes of a committed transaction survive crash. 	
	 
	 For the second case, if the dirty write is written to the disk in yet uncommitted transaction at the time of subsequent crash, Without a records of these changes, there would be no way to undo the changes.
	 
	 They are sufficient for durability because they support redoing modifications and ensure all actions of committed transactions survive system crashes and media failures.
	 
\end{enumerate}

\section{Question 2}
\begin{enumerate}
	\item

\end{enumerate}

\end{document}
