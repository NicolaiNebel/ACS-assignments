\documentclass[11pt]{article}

\usepackage[utf8]{inputenc}
\usepackage[T1]{fontenc}
\usepackage[head=26pt, a4paper, margin=1.2in, top=1.4in, bottom=1.75in]{geometry}
\usepackage{fancyhdr}
\usepackage{lastpage}
\usepackage[hidelinks, colorlinks, urlcolor=blue, linkcolor=black,citecolor=magenta]
{hyperref}
\usepackage{amsmath}
\usepackage{amsthm}
\usepackage{amssymb}
\usepackage{graphicx}
\usepackage{float}
\usepackage{listings}
\usepackage{mathtools}
\usepackage{enumitem}
\usepackage[table,xcdraw]{xcolor}

\usepackage{indentfirst}
\usepackage{a4wide}
\usepackage{color}
\usepackage{lipsum}
\usepackage{multicol}
\usepackage{tikz}
\usetikzlibrary{arrows,shapes,positioning}

% ---------------- Page and margin/header/footer Setup -----------------
\pagestyle{fancy}
%\fancyhf{} % Clears header and footer
\fancyhead{}
\fancyfoot{}
\lhead{ACS --- Assignment 4}
\rhead{DIKU}
\lfoot{Page \thepage\ of \pageref{LastPage}}
\rfoot{Nicolai Jørgensen \\ Yiran Zhang}
\renewcommand{\headrulewidth}{0.4pt}
\renewcommand{\footrulewidth}{0.4pt}
% ----------------------------------------------------------------------

\newtheorem{mythm}{Theorem}
\newtheorem{mydef}{Definition}

\DeclarePairedDelimiter{\ceil}{\lceil}{\rceil}
\newcommand\numberthis{\addtocounter{equation}{1}\tag{\theequation}}

\newcommand{\HRule}{\rule{\linewidth}{0.5mm}}

\title          {Assignment 4}
\author         {Nicolai Jørgensen and Yiran Zhang}

\begin{document}

\maketitle
\newpage

\section{Reliability}

We assume that $p$ is the probability of failure over some amount of time and we
are computing the probability of the system being connected after one such
amount of time.

\begin{enumerate}
  \item
    The system will become disconnected if one of the two wires fail. That means
    the failure probability is $Pr(X \leq 0)$, where $X ~ binom(2,p)$.
  \item
    In this case, the system will become disconnected if two of the three links
    fail, so the failure probability is $Pr(Y \leq 1)$ where $Y ~ binom(3,p)$.
  \item
    We compute the two probabilities:
    $$ Pr(X \leq 0) = { 2 \choose 0 } 0.000001^0(1-0.000001)^2 \approx 0.99999800 $$
    $$ Pr(X \leq 1) = \sum_{i=0}^1 { 3 \choose i }0.0001^i(1-0.0001)^{(3-i)} \approx 0.99999997 $$
    From this we conclude that the town council should buy the low-reliability links.
\end{enumerate}

\section{ARIES}

\begin{enumerate}
  \item
    Here is the dirty page table computed in the analysis phase:

    \begin{center}
      \begin{tabular}{ll}
      \rowcolor[HTML]{C0C0C0} 
      PageID & RecLSN \\
      P2     & 3      \\
      P1     & 4      \\
      P5     & 5      \\
      P3     & 6     
      \end{tabular}
    \end{center}

    And here is the transaction table:

    \begin{center}
      \centering
      \begin{tabular}{lll}
      \rowcolor[HTML]{C0C0C0} 
      TransID                    & Status & LastLSN \\
      \cellcolor[HTML]{6434FC}T1 & Active & 4       \\
      \cellcolor[HTML]{67FD9A}T2 & Active & 9      
      \end{tabular}
    \end{center}
  \item
    The set of winner transactions is ${T3}$ since it is the only one that finished.

    The set of loser transactions is ${T1,T2}$ since these did not finish before the crash.
  \item
    The redo phase starts at the minimum \verb|recLSN| in the dirty page table. This means 
    \verb|LSN 3|.

    The undo phase ends at the oldest \verb|LSN| of the transactions in the
    loser set. That would mean \verb|LSN 3|, since that is the first \verb|LSN|
    associated with \verb|T1|.
  \item
    The set of log records that may ause pages to be rewritten during the redo phase will
    consist of all \verb|update| or \verb|CLR| records after \verb|LSN 3|, where the redo phase starts.

    This means that the set is ${3,4,5,6,8,9}$.
  \item
    The set of log records to undo is the set of updates of the loser transactions. That means LSNs ${9,8,5,4,3}$.
  \item
    This is what is appended to the log after the recovery procedure is completed
    following a crash after \verb|LSN 10|.
    \begin{verbatim}
      LSN   LAST_LSN   TRAN_ID   TYPE             undoNextLSN  PAGE_ID
      ---   --------   -------   ----             -----------  -------
      11    9          T2        ABORT                         -
      12    4          T1        ABORT                         -
      13    11         T2        CLR: Undo LSN 9  8
      14    13         T2        CLR: Undo LSN 8  5
      15    14         T2        CLR: Undo LSN 5  -
      16    15         T2        end
      17    12         T1        CLR: Undo LSN 4  3
      18    17         T1        CLR: Undo LSN 3  -
      19    18         T1        end
    \end{verbatim}
\end{enumerate}
\end{document}
